La leggibilità è una dimensione di qualità che misura la facilità e 
l'immediatezza di comprensione di una determinata rappresentazione dei dati. 
Una rappresentazione dalla buona leggibilità si presta ad analisi (almeno 
sommarie) da parte degli utilizzatori dei dati, che possono in questo modo 
trarne informazioni utili ai loro processi decisionali.

I dataset grezzi scelti per il progetto presentano diverse problematiche sotto 
questo punto di vista. In primo luogo, le unità di misura utilizzate sono 
quelle del sistema anglosassone, che risultano di difficile lettura al di fuori 
degli Stati Uniti. Inoltre, vi sono attributi che impiegano codici alfanumerici 
impossibili da interpretare senza consultare la documentazione.

Nel seguito discutiamo le modifiche che abbiamo apportato ai vari dataset per 
migliorarne la leggibilità.

\begin{itemize}

    \item \textsc{Weather}: 
        
        \begin{itemize}
            \item Questo dataset utilizza il carattere \texttt{M} per indicare 
            i valori mancanti. È stato scelto di sostituirlo con il valore 
            \texttt{null} poiché più esplicito e poiché riconosciuto 
            automaticamente anche dai software di analisi dei dati e dai DBMS.
            
            \item \texttt{date}: inizialmente l'attributo rappresentava una 
                data attraverso il pattern \textit{yyyyMMdd}, per migliorarne 
                la leggibilità è stato trasformato nella forma 
                \textit{dd/MM/yyyy}. Ad esempio la data \textit{20070501} è 
                stata modificata in \textit{01/05/2007}.
        
            \item \texttt{t\_max}, \texttt{t\_min},  \texttt{t\_avg}, 
            \texttt{depart}, \texttt{dew\_point}, \texttt{wet\_bulb}, 
            \texttt{heat}, \texttt{cool}: i valori di temperatura sono stati 
            convertiti da gradi Fahrenheit (\si{\degree F}) a gradi Celsius 
            (\si{\celsius}).

           	\item \texttt{snow\_depth}, \texttt{snow\_water}, 
           	\texttt{snow\_fall}, \texttt{precip\_total}: i valori sono stati 
           	convertiti da pollici (inch) a millimetri (mm).

            \item \texttt{stn\_pressure}, \texttt{sea\_level}: i valori di 
            pressione sono stati convertiti da pollici di mercurio (inHg) a 
            millimetri di mercurio (mmHg).

            \item \texttt{result\_speed}, \texttt{avg\_speed}, 
            \texttt{max5\_speed}, \texttt{max2\_speed}: i valori di velocità 
            sono stati convertiti da miglia orarie (mph) a chilometri orari 
            (\si[per-mode=symbol]{\km\per\hour}).
            
            \item \texttt{code\_sum}: questo attributo corrisponde ad una lista 
            di codici che possono essere assegnati a ciascuna rilevazione, 
            indicanti i diversi tipi di precipitazioni e altri fenomeni
            meteorologici. I codici alfanumerici originali sono stati 
            trasformati nei corrispondenti nomi in linguaggio naturale (ad es. 
            pioggia, neve, nebbia, \dots). Inoltre, dato che a ciascuna 
            rilevazione può essere assegnato più di un codice, abbiamo 
            trasformato la singola colonna del dataset originale in una colonna 
            per ciascun codice possibile, in cui un valore booleano indica la 
            presenza o l'assenza della rispettiva condizione meteo.
        \end{itemize}
	
	\item \textsc{Station}:
	\begin{itemize}
    	\item \texttt{ground\_height}, \texttt{station\_height}, 
    	\texttt{barometer}: i valori di altitudine sono stati convertiti da
    	piedi (feet) a metri (m).
	\end{itemize}
		
	\item \textsc{WNV Mosquito}:
	
	\begin{itemize}
		
		\item \texttt{trap}: nel dataset le trappole vengono contrassegnate 
		dalla lettera \textit{T} seguita da tre cifre. Trappole "satellite" 
		sono spesso installate vicino ad una trappola principale per potenziare 
		la sorveglianza. Per questo motivo sono indicate dallo stesso codice 
		della trappola principale seguito da una lettera. Ad esempio T220A è la 
		prima trappola satellite per T220. Essendo questo di difficile lettura 
		sono state aggiunte due colonne: \texttt{main\_trap} che riporta il 
		codice della trappola principale e \texttt{sub\_trap} che contiene nel 
		caso delle trappole satellite la lettera ad essa associata.
		
		\item \texttt{result}: inizialmente la colonna conteneva dei valori  
		testuali, \textit{positive} e \textit{negative}, che sono stati 
		trasformati in valori booleani.			
	\end{itemize}
\end{itemize}
