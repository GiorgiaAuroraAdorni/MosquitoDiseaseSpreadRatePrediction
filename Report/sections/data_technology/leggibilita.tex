La leggibilità è una dimensione di qualità utile per permettere a chi vuole 
utilizzare una collezione di dati di analizzare almeno sommariamente le 
informazioni in modo da poter prendere determinate decisioni riguardanti 
l'utilizzo dei dati, come ad esempio quali informazioni potrebbero essere più 
significative per un modello predittivo di machine learning.

I dataset scelti per il progetto utilizzano unità di misura del sistema 
anglosassone, dunque di difficile lettura per chi utilizza il sistema 
internazionale, oppure vengono impiegate sigle alfanumeriche per indicare un 
determinato tipo di valori. 

Di seguito verranno discusse le modifiche apportate agli attributi dei vari 
dataset per migliorarne la leggibilità.

\begin{itemize}

    \item \textsc{Weather}: Il dataset utilizza il valore \textit{M} per 
        indicare un valore mancante per un attributo, in questo caso è stato 
        scelto di sostituire a \textit{M} il valore \textit{null} poichè viene 
        riconosciuto correttamente come valore mancante dai software per 
        l'import/export dei dati e dai DBMS.
        
        \begin{itemize}
            
            \item \texttt{date}: inizialmente l'attributo rappresentava una 
                data attraverso il pattern \textit{yyyyMMdd}, per migliorarne 
                la leggibilità è stato cambiato il pattern e utilizzato nella 
                forma \textit{dd/MM/yyyy}. Ad esempio la data 
                \textit{20070501} è stata modificata in \textit{01/05/2007}.
        
            \item \texttt{t\_max}, \texttt{t\_min},  \texttt{t\_avg}, \texttt{depart}, 
            	\texttt{dew\_point}, \texttt{wet\_bulb}, \texttt{heat}, 
            	\texttt{cool}: i valori di temperatura sono stati convertiti da 
            	gradi Fahrenheit (\si{\degree F}) a gradi Celsius 
            	(\si{\celsius}).

           	\item \texttt{snow\_depth}, \texttt{snow\_water}, \texttt{snow\_fall}, 
           		\texttt{precip\_total}: i valori sono stati convertiti da pollici
           		(inch) a millimetri (mm).

            \item \texttt{stn\_pressure}, \texttt{sea\_level}: i valori di 
                pressione sono stati convertiti da pollici di mercurio (inHg) a 
                millimetri di mercurio (mmHg).

            \item \texttt{result\_speed}, \texttt{avg\_speed}, \texttt{max5\_speed}, 
            	\texttt{max2\_speed}: i valori di velocità sono stati convertiti da
            	miglia orarie (mph) a chilometri orari 
            	(\si[per-mode=symbol]{\km\per\hour}).
            
            \item \texttt{code\_sum}: questo attributo corrisponde ad una lista di 
            	codici che possono essere assegnati a ciascuna rilevazione, 
            	corrispondenti ai diversi tipi di precipitazioni e altri fenomeni
            	meteorologici (ad es. pioggia, neve, nebbia, \dots). Dato che a 
            	ciascuna rilevazione può essere assegnato più di un codice, è stato
            	necessario effettuare un parsing dei valori, creando una colonna per
            	ogni codice utilizzato.
        \end{itemize}
	
	\item \textsc{WNV Mosquito}:
	
	\begin{itemize}
		
		\item \texttt{trap}: nel dataset le trappole vengono contrassegnate 
		dalla lettera \textit{T} seguita da tre cifre. Trappole "satellite" 
		sono spesso installate vicino ad una trappola principale per potenziare 
		la sorveglianza. Per questo motivo sono indicate dallo stesso codice 
		della trappola principale seguito da una lettera. Ad esempio T220A è la 
		prima trappola satellite per T220. Essendo questo di difficile lettura 
		sono state aggiunte due colonne: \texttt{main\_trap} che riporta il 
		codice della trappola principale e \texttt{sub\_trap} che contiene nel 
		caso delle trappole satellite la lettera ad essa associata.
		
		\item \texttt{result}: inizialmente la colonna conteneva dei valori  
		testuali, \textit{positive} e \textit{negative}, che sono stati 
		trasformati in valori booleani.  
		
		\item \texttt{number\_of\_mosquitoes}: \textbf{FIXME questo era già 
		così nel dataset originale, non l'abbiamo fatto noi.} nel caso di 
		numero di zanzare superiore a 50, questo viene diviso in un altro 
		record con gli stessi attributi in modo tale che il numero di zanzare 
		sia limitato a 50.  
			
	\end{itemize}
	
\end{itemize}

