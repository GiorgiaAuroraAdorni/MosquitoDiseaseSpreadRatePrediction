\section{Acquisizione dei dataset}
Il dataset \textsc{WNV Mosquito} contiene la posizione di ogni trappola 
installata nella città attraverso il programma di Salute Ambientale del 
Dipartimento di Sanità Pubblica di Chicago negli anni tra il 2007 e il 2018. 
Contiene inoltre la specie e il numero di zanzare trovate in ciascuna trappola 
ogni settimana, nonchè i risultati dei test per il virus del Nilo occidentale. 
È stato reperito dal sito Chicago Data 
Portal\footnote{\url{https://data.cityofchicago.org/Health-Human-Services/West-Nile-Virus-WNV-Mosquito-Test-Results/jqe8-8r6s/data}}.
\\

I dataset \textsc{Weather} e \textsc{Stations} provengono invece dal database 
NOAA Quality Controlled Local Climatological Data (QCLCD) disponibile sul sito 
dalla National Oceanic and Atmospheric 
Administration\footnote{\url{https://www.ncdc.noaa.gov/data-access/land-based-station-data/land-based-datasets/quality-controlled-local-climatological-data-qclcd}}.
I dati grezzi sono composti da un file CSV per ciascun mese nel periodo 
2007--2017, al cui interno sono presenti le rilevazioni di tutte le stazioni 
meteorologiche degli Stati Uniti.

Come prima cosa abbiamo scritto una serie di script Bash con i quali estrarre 
esclusivamente le stazioni meteorologiche nei pressi della città di Chicago e 
aggregarle in un unico file contenente l'intero arco temporale di interesse. 
Attraverso un ulteriore script Bash abbiamo analizzato le colonne presenti nei 
file CSV, per accertarci che la loro struttura fosse rimasta costante negli 
anni.

Si può notare che vi è una discrepanza nei periodi temporali coperti dalle due 
fonti, in quanto i dati QCLCD 2018 non sono ad oggi disponibili online. Come 
verrà descritto più in dettaglio nella Sezione~\ref{sec:data-fusion}, 
questo ci ha costretto a limitarci al periodo 2007--2017, escludendo quindi 
l'anno 2018 dal dataset WNV Mosquito.

\section{Descrizione dei tre dataset}
Vengono di seguito descritti gli attributi di ciascun dataset.

\begin{longtable}{lll}
	\toprule
	\textbf{Attributo} \quad & \textbf{Descrizione} & \textbf{Valori} \\
	\midrule
	\endfirsthead
	\multicolumn{3}{l}{\footnotesize\itshape Continua dalla pagina precedente} \\
	\toprule
	\textbf{Attributo} \quad & \textbf{Descrizione} & \textbf{Valori} \\			
	\endhead
	\multicolumn{3}{l}{\footnotesize\itshape Continua nella pagina 
		seguente} \\
	\endfoot
	\endlastfoot
	
	\multirow{2}{*}{season\_year} & \multirow{2}{*}{anno del test} &  categorico \\
	& &{2007} -- {2017}       	\\ \hline
	\multirow{2}{*}{week}					&\multirow{2}{*}{settimana dell'anno del test}  & numerico    \\ 
	& & {1} -- {52} ({20} -- {40})\\\hline
	test\_id			& id del record &  numerico        	\\\hline 
	\multirow{3}{*}{block}	& \multirow{3}{*}{indirizzo della trappola} & xxXX 
	= numero civico \\
	& & STREET = nome della via	\\
	& & xxXX STREET \\ \hline
	\multirow{2}{*}{trap}	&\multirow{2}{*}{id della trappola} &  
	identificatore        	\\ 
	&& T001X -- T925X\\\hline
	\multirow{4}{*}{trap\_type}				& \multirow{4}{*}{tipologia di trappola} &     OVI\\
	&& CVC\\
	&& GRAVID\\
	&& SENTINEL\\\hline
	{test\_date}	&{data del test}   &   mm/dd/yyyy hh:mm:ss P \\\hline
	\multirow{2}{*}{number\_of\_mosquitoes}	& \multirow{2}{*}{numero di zanzare catturate} &  numerico        	\\ 
	& & {1} -- {50} \\ \hline
	\multirow{2}{*}{result}	& \multirow{2}{*}{risultato del test}   &  positive = WNV presente \\
	& & negative = WNV non presente  	\\ \hline
	\multirow{8}{*}{species}	& \multirow{8}{*}{specie di zanzare}	 & CULEX PIPIENS/RESTAUANS \\
	& & CULEX PIPIENS \\
	& & CULEX RESTAUANS \\
	& & CULEX ERRATICUS \\
	& & CULEX SALINARIUS \\
	& & CULEX TARSALIS \\
	& & CULEX TERRITANS \\
	& & UNSPECIFIED CULEX  \\	\hline
	\multirow{2}{*}{latitude}				& \multirow{2}{*}{latitudine dell'indirizzo}	 &  decimale       \\ 
	& & {40.00} -- {42.00}  \\ \hline
	\multirow{2}{*}{longitude}				& \multirow{2}{*}{longitudine dell'indirizzo} &  decimale      	\\ 
	&& {-87.00} -- {-88.00} \\
	\bottomrule
\end{longtable}
\captionof{table}{Attributi del dataset \textsc{WNV Mosquito}}
\label{tab:attributi wnv}

$\\$
\begin{longtable}{lll}
	\toprule
	\textbf{Attributo} \quad & \textbf{Descrizione} & \textbf{Valori} \\
	\midrule
	\endfirsthead
	\multicolumn{3}{l}{\footnotesize\itshape Continua dalla pagina precedente} \\
	\toprule
	\textbf{Attributo} \quad & \textbf{Descrizione} & \textbf{Valori} \\			
	\endhead
	\multicolumn{3}{l}{\footnotesize\itshape Continua nella pagina seguente} \\
	\endfoot
	\endlastfoot
	wban & id della stazione	 & numerico  \\ 
	\hline	
	{year\_month\_day} & {data della rilevazione}	 & yyyymmdd \\ \hline	
	\multirow{2}{*}{t\_max}				& temperatura massima	 & numerico		   	\\
	&giornaliera (\degree F) &(*) estremo del mese \\ \hline
	\multirow{2}{*}{t\_max\_flag}		& \multirow{2}{*}{qualità t\_max} &  blank = valore non sospetto \\
	& & s = valore sospetto			 \\ \hline	
	\multirow{2}{*}{t\_min}				& temperatura minima  & numerico		   	\\
	&giornaliera (\degree F) &(*) estremo del mese \\ \hline
	\multirow{2}{*}{t\_min\_flag}		& \multirow{2}{*}{qualità t\_min} &  blank = valore non sospetto \\
	& & s = valore sospetto			 \\ \hline		
	\multirow{2}{*}{t\_avg}				& temperatura media  & \multirow{2}{*}{numerico}	   	\\ 
	&giornaliera (\degree F)&	\\\hline
	\multirow{2}{*}{t\_avg\_flag}		& \multirow{2}{*}{qualità t\_avg} & blank = valore non sospetto \\
	& & s = valore sospetto			 \\ 
	\hline		
	\multirow{2}{*}{depart}				& scostamento dalla  	 & 	\multirow{2}{*}{numerico}   	\\ 
	& temperatura normale (\degree F)&	\\
	\hline
	\multirow{2}{*}{depart\_flag}		& \multirow{2}{*}{qualità depart} &  blank = valore non sospetto \\
	& & s = valore sospetto			 \\ 
	\hline
	\multirow{3}{*}{dew\_point}	& temperatura, a pressione  	 & 	\multirow{3}{*}{numerico}	   	\\ 
	&  costante, in cui l'aria  &\\
	& diventa satura (\degree F) &		\\
	\hline
	\multirow{2}{*}{dew\_point\_flag}		& \multirow{2}{*}{qualità dew\_point} & blank = valore non sospetto \\
	& & s = valore sospetto			 \\ 
	\hline
	\multirow{4}{*}{wet\_bulb}	& temperatura di equilibrio     & 	\multirow{4}{*}{numerico}	   	\\ 
	& di scambio convettivo e di   &\\
	& massa d'aria in moto & \\
	& turbolento dell'acqua (\degree F)&\\
	\hline
	\multirow{2}{*}{wet\_bulb\_flag}		& \multirow{2}{*}{qualità wet\_bulb} &  blank = valore non sospetto \\
	& & s = valore sospetto			 \\ 
	\hline
	\multirow{5}{*}{heat}	& numero di gradi in cui la   & 	\multirow{5}{*}{numerico}	   	\\ 
	& temperatura media giornaliera  &\\
	& è inferiore a 65 \degree F (temperatura   &\\
	& al di sotto della quale gli edifici  & \\
	& devono essere riscaldati) &\\
	\hline
	\multirow{2}{*}{heat\_flag}		& \multirow{2}{*}{qualità heat} &  blank = valore non sospetto \\
	& & s = valore sospetto			 \\ 
	\hline		
	\multirow{5}{*}{cool}	& numero di gradi in cui la    & 	\multirow{5}{*}{numerico}	   	\\ 
	& temperatura media giornaliera   &\\
	& è superiore a 65 \degree F (temperatura  &\\
	& al di sopra della quale gli edifici  &\\
	& devono essere raffreddati) &\\
	\hline
	\multirow{2}{*}{cool\_flag}		& \multirow{2}{*}{qualità cool} &  blank = valore non sospetto \\
	& & s = valore sospetto			 \\ 
	\hline		
	sunrise				& ora dell'alba (LST)	 & hhmm		   	\\ 
	\hline
	\multirow{2}{*}{sunrise\_flag}		& \multirow{2}{*}{qualità sunrise} &  blank = valore non sospetto \\
	& & s = valore sospetto			 \\ 
	\hline	
	sunset				& ora del tramonto (LST) & hhmm		   	\\ 
	\hline
	\multirow{2}{*}{sunset\_flag}		& \multirow{2}{*}{qualità sunset} &  blank = valore non sospetto \\
	& & s = valore sospetto			 \\ 
	\hline	
	\multirow{15}{*}{code\_sum}			& \multirow{15}{*}{ }	 & 	 BR = foschia \\
	& & DU = polvere\\
	& & DZ = pioggerella \\
	& & FC = nubi a imbuto \\
	& & FG = nebbia \\
	& & FU = fumo \\
	& lista di condizioni & GR = grandine \\
	& meteorologiche giornaliere & GS = piccola grandine\\
	& & HZ = foschia \\
	& & PL = pioggia gelata\\
	& & RA = pioggia \\
	& & SN = neve \\
	& & SQ = raffiche di vento\\
	& & TS = temporale \\
	& & UP = sconosciuto \\	\hline	   	
	\multirow{2}{*}{code\_sum\_flag}		& \multirow{2}{*}{qualità code\_sum} &  blank = valore non sospetto \\
	& & s = valore sospetto			 \\ 
	\hline	
	\multirow{2}{*}{depth}	& quantità di neve depositata  	 & \multirow{2}{*}{numerico}		 	\\ 
	& a terra (millimetri) & \\
	\hline
	\multirow{2}{*}{depth\_flag}		& \multirow{2}{*}{qualità depth} &  blank = valore non sospetto \\
	& & s = valore sospetto			 \\ 
	\hline
	\multirow{3}{*}{water1}	& quantità di acqua ottenuta  	 & \multirow{3}{*}{numerico}		 	\\ 
	& sciogliendo la neve a terra  & \\
	&  (millimetri) & \\
	\hline
	\multirow{2}{*}{water1\_flag}		& \multirow{2}{*}{qualità water1} &  blank = valore non sospetto \\
	& & s = valore sospetto			 \\ 
	\hline	
	\multirow{2}{*}{snow\_fall}			& quantità giornaliera di 	 & \multirow{2}{*}{decimale}				\\ 
	& neve caduta (inch) &\\
	\hline
	\multirow{2}{*}{snow\_fall\_flag}		& \multirow{2}{*}{qualità snow\_fall} &  blank = valore non sospetto \\
	& & s = valore sospetto			 \\ 
	\hline
	\multirow{2}{*}{precip\_total}		& quantità di precipitazioni 	 & 
	\multirow{2}{*}{decimale}			\\ 
	& giornaliere (inch) &\\\hline	
	\multirow{2}{*}{precip\_total\_flag}		& \multirow{2}{*}{qualità precip\_total} &  blank = valore non sospetto \\
	& & s = valore sospetto			 \\ 
	\hline	
	\multirow{2}{*}{stn\_pressure}		& pressione media giornaliera	 & 	
	\multirow{2}{*}{decimale}		\\ 
	&   (pollici di mercurio) &\\
	\hline	
	\multirow{2}{*}{stn\_pressure\_flag}		& \multirow{2}{*}{qualità stn\_pressure} &  blank = valore non sospetto \\
	& & s = valore sospetto			 \\ 
	\hline	
	\multirow{3}{*}{sea\_level}			& pressione media giornaliera 	 & 	
	\multirow{3}{*}{decimale}		\\ 
	& a livello del mare &\\
	& (pollici di mercurio)&\\
	\hline	
	\multirow{2}{*}{sea\_level\_flag}		& \multirow{2}{*}{qualità sea\_level} &  blank = valore non sospetto \\
	& & s = valore sospetto			 \\ 
	\hline
	\multirow{2}{*}{result\_speed}		& velocità di punta giornaliera 	 & 	\multirow{2}{*}{decimale}	\\ 
	& del vento (miglia all'ora) &\\
	\hline	
	\multirow{2}{*}{result\_speed\_flag}		& \multirow{2}{*}{qualità result\_speed} &  blank = valore non sospetto \\
	& & s = valore sospetto			 \\ 
	\hline	
	\multirow{2}{*}{result\_dir}			& direzione giornaliera del vento  & 	\multirow{2}{*}{numerico}	\\ 
	& durante la velocità di punta &\\
	\hline	
	\multirow{2}{*}{result\_dir\_flag}		& \multirow{2}{*}{qualità result\_dir} &  blank = valore non sospetto \\
	& & s = valore sospetto			 \\ 
	\hline		
	\multirow{2}{*}{avg\_speed}		& velocità media giornaliera 	 & 	\multirow{2}{*}{decimale}	\\ 
	& del vento (miglia all'ora) &\\
	\hline	
	\multirow{2}{*}{avg\_speed\_flag}		& \multirow{2}{*}{qualità avg\_speed} &  blank = valore non sospetto \\
	& & s = valore sospetto			 \\ 
	\hline		
	\multirow{3}{*}{max5\_speed}		& velocità massima giornaliera 	 & 	\multirow{3}{*}{numerico}	\\ 
	& raggiunta dal vento per 5 &\\
	& minuti (miglia all'ora) &\\
	\hline	
	\multirow{2}{*}{max5\_speed\_flag}		& \multirow{2}{*}{qualità max5\_speed} &  blank = valore non sospetto \\
	& & s = valore sospetto			 \\ 
	\hline		
	\\
	\hline
	\multirow{3}{*}{max5\_dir}		& direzione giornaliera del vento  & 	\multirow{3}{*}{numerico}	\\ 
	& durante la massima velocità  &\\
	& raggiunta per 5 minuti &\\
	\hline
	\multirow{2}{*}{max5\_dir\_flag}		& \multirow{2}{*}{qualità max5\_dir} &  blank = valore non sospetto \\
	& & s = valore sospetto			 \\ 
	\hline	
	\multirow{3}{*}{max2\_speed}		& velocità massima giornaliera 	 & 	\multirow{3}{*}{numerico}	\\ 
	& raggiunta dal vento per &\\
	& 2 minuti (miglia all'ora) &\\
	\hline		
	\multirow{2}{*}{max2\_speed\_flag}		& \multirow{2}{*}{qualità max2\_speed} &  blank = valore non sospetto \\
	& & s = valore sospetto			 \\ 
	\hline			
	\multirow{3}{*}{max2\_dir}		& direzione giornaliera del vento  & 	\multirow{3}{*}{numerico}	\\ 
	& durante la massima velocità  &\\
	& raggiunta per 2 minuti &\\
	\hline
	\multirow{2}{*}{max2\_dir\_flag}		& \multirow{2}{*}{qualità max2\_dir} &  blank = valore non sospetto \\
	& & s = valore sospetto			 \\ 
	\bottomrule
\end{longtable}
\captionof{table}{Attributi del dataset \textsc{Weather}}
\label{tab:attributi weather}

$\\$
\begin{longtable}{lll}
	\toprule
	\textbf{Attributo} \quad & \textbf{Descrizione} & \textbf{Valori} \\
	\midrule
	\endfirsthead
	\multicolumn{3}{l}{\footnotesize\itshape Continua dalla pagina precedente} \\
	\toprule
	\textbf{Attributo} \quad & \textbf{Descrizione} & \textbf{Valori} \\			
	\endhead
	\multicolumn{3}{l}{\footnotesize\itshape Continua nella pagina seguente} \\
	\endfoot
	\endlastfoot
	wban	&   codice WBAN della stazione & numerico   			\\ \hline	
	wmo		& 	codice WMO della stazione  & numerico				\\ \hline	
	\multirow{2}{*}{callsign}	& 	codice IATA dell'aeroporto  & 
	\multirow{2}{*}{categorico}	    \\
	& in cui si trova la stazione & \\ \hline
	climate\_division\_code			& codici che identificano la & numerico	\\	
	climate\_division\_state\_code	& regione climatica a cui    & numerico	\\ 
	climate\_division\_station\_code& la stazione appartiene     & numerico	\\ 
	\hline	
	name	& 	nome della stazione	 & 	categorico	     	\\ \hline
	state		& 	sigla dello stato 			& categorico		\\ \hline	
	location	& 	posizione della stazione 	& categorico     	\\ \hline
	\multirow{2}{*}{latitude}	& \multirow{2}{*}{latitudine della stazione}	
	&  decimale    \\ 
	& & {40.00} -- {42.00}  \\ \hline
	\multirow{2}{*}{longitude}	& \multirow{2}{*}{longitudine della stazione} 
	&  decimale     \\ 
	& & {-87.00} -- {-88.00} \\ \hline \\ \hline
	\multirow{2}{*}{ground\_height}	& 	altitudine del terreno sul  & 
	\multirow{2}{*}{numerico}	\\
	& livello del mare (piedi) &\\ \hline		
	\multirow{2}{*}{station\_height}	& 	altitudine della stazione sul  &   
	\multirow{2}{*}{numerico} 	\\
	& livello del mare (piedi) & \\ \hline
	\multirow{2}{*}{barometer}			& 	altitudine del barometro sul  & 	
	\multirow{2}{*}{numerico}		\\
	& livello del mare (piedi) &\\ \hline	
	time\_zone		& 	fuso orario della stazione & numerico		  \\
	\bottomrule
\end{longtable}
\captionof{table}{Attributi del dataset \textsc{Stations}}
\label{tab:attributi stations}


