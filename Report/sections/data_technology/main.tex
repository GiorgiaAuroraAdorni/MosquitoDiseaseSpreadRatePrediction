\setcounter{chapter}{0}

\part{Data Technology}

\chapter{Dataset}
\sections{data_technology/dataset}

\chapter{Data Quality}
\label{chap:data-quality}
Il processo di analisi a cui abbiamo sottoposto i dataset era finalizzato 
all'esplorazione dei dati e all'incremento della loro qualità. In particolare, 
in funzione della natura dei dati e del loro scopo, \textbf{FIXME siamo ricorsi 
all'utilizzo} delle dimensioni di completezza, consistenza e leggibilità.

% Metriche dei dataset analizzati singolarmente

\section{Completezza}
\sections{data_technology/completezza}

\section{Consistenza di chiave}
\sections{data_technology/consistenza}

\section{Leggibilità}
\sections{data_technology/leggibilita}

\section{Deduplicazione}
\sections{data_technology/deduplicazione}

\section{Acquisizione di nuovi dati}

\chapter{Data Integration}
\label{chap:data-integration}

\sections{data_technology/integrazione}

% record linkage


\chapter{Analisi di almeno 2 dimensioni di qualità e relative metriche delle features successivamente utilizzate}

\chapter{Analisi descrittive dei dati integrati}

* FIXME mappa con i block colorati per stazione meteo più vicina

% fusion
