\setcounter{chapter}{0}

\part{Data Technology}

\chapter{Descrizione delle modalità di scelta e acquisizione delle sorgenti dati e scelta del modello di descrizione del dataset}

\chapter{Data Quality}


Il processo di analisi a cui abbiamo sottoposti i dataset era finalizzato all'esplorazione dei dati e all'incremento della loro qualità. In particolare, in funzione della natura dei dati e del loro scopo, siamo ricorsi all'utilizzo delle dimensioni di completezza, consistenza e leggibilità.

% Metriche dei dataset analizzati singolarmente

\section{Completezza}
\sections{data_technology/completezza}

\section{Consistenza di chiave}
\sections{data_technology/consistenza}

\section{Leggibilità}
\sections{data_technology/leggibilita}

\section{Deduplicazione}


\section{Acquisizione di nuovi dati}
% Geocode di google maps

\chapter{Processo di integrazione dei dati ed eventuali problemi riscontrati includendo le eterogeneità riscontrate}

% record linkage


\chapter{Analisi di almeno 2 dimensioni di qualità e relative metriche delle features successivamente utilizzate}
\chapter{Analisi descrittive dei dati integrati}

% fusion
