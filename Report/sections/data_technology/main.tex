\setcounter{chapter}{0}

\part{Data Technology}

\chapter{Dataset}
\sections{data_technology/dataset}

\chapter{Data Quality}
Il processo di analisi a cui abbiamo sottoposto i dataset era finalizzato 
all'esplorazione dei dati e all'incremento della loro qualità. In particolare, 
in funzione della natura dei dati e del loro scopo, \textbf{FIXME siamo ricorsi 
all'utilizzo} delle dimensioni di completezza, consistenza e leggibilità.

% Metriche dei dataset analizzati singolarmente

\section{Completezza}
\sections{data_technology/completezza}

\section{Consistenza di chiave}
\sections{data_technology/consistenza}

\section{Leggibilità}
\sections{data_technology/leggibilita}

\section{Acquisizione di nuovi dati}

\chapter{Processo di integrazione dei dati ed eventuali problemi riscontrati includendo le eterogeneità riscontrate}

% record linkage
\section{Record Linkage}
\sections{data_technology/record_linkage}

\section{Deduplication}
\sections{data_technology/deduplicazione}

%fusion
\section{Data Fusion}
\sections{data_technology/data_fusion}

\chapter{Analisi di almeno 2 dimensioni di qualità e relative metriche delle features successivamente utilizzate}
\chapter{Analisi descrittive dei dati integrati}
