La prima dimensione di qualità analizzata è la completezza, ovvero la copertura con cui vengono rappresentati gli insiemi di dati.
In particolare sono state distinte la completezza di tupla, che specifica la presenza di valori nulli in una tupla, la completezza degli attributi ovvero la mancanza di valori in una colonna di attributo e infine la completezza della tabella, ovvero il numero totale di valori non nulli in una tabella.\\

Le metriche utilizzate per la misurazione della completezza sono rispettivamente:
\begin{itemize}
		\item[-] Completezza di tupla: il rapporto tra il numero di valori nulli nella tupla e la cardinalità della tupla (ovvero il numero di attributi);
		\item[-] Completezza di attributo: il rapporto tra il numero di valori nulli della colonna e la cardinalità della colonna (ovvero il numero di tuple);
		\item[-] Completezza di tabella: il rapporto tra il numero di valori nulli nella tabella e la cardinalità della tabella (data dal prodotto cartesiano tra righe e colonne).
\end{itemize}

% Motivo della scelta della dimensione di qualità
% Con cosa è stato effettuato il conteggio dei valori null

La completezza totale dei quattro dataset è del 71,43\%, mentre la completezza delle tuple è solo del 34,96\%.

\begin{figure}[H]
	\centering
	\begin{tabular}{lcccc}
		\toprule
		\textbf{Dataset} \quad & \textbf{Completezza di tupla} & \textbf{Completezza dataset} \\
		\midrule
		WNV Mosquito &		84,89\%  	& 97,48\%  \\ 
		Weather 	 &		 0,00\% 	& 86,09\%  \\ 
		Stations 	 &		20,00\% 	& 82,67\%  \\ 
		\midrule
		Totale 		 &	    34,96\%     & 88,75\%  \\
		\bottomrule
	\end{tabular}
	\captionof{table}{Analisi di completezza sui tre dataset}
	\label{tab:completezza totale}
\end{figure}

Vengono in seguito mostrate le analisi relative ai dati di ciascun dataset.

\subsubsection*{WNV Mosquito}
\addcontentsline{toc}{subsubsection}{WNV Mosquito}
Il dataset \textsc{WNV Mosquito} presenta una completezza del 97,48\%. Esso risulta essere il più completo tra i dataset presentati. In particolare solo due attributi su dodici, \textit{latitude} e \textit{longitude}, presentano dei valori mancanti.\\
In tabella \ref{tab:completezza wnv} sono presentate le analisi di completezza relative a ciascun attributo e la percentuale della completezza di tupla e di schema.

\begin{figure}[H]
	\centering
	\begin{tabular}{lcccc}
		\toprule
		\textbf{Attributo} \quad & \textbf{Istanze} & \textbf{Valori nulli} & 
		\textbf{Completezza} \\
		\midrule
		season\_year			&			27196 &  0        &  100\%   	\\ 
		week					&			27196 &  0        &  100\%   	\\ 
		test\_id				&			27196 &  0        &  100\%   	\\ 
		block					&			27196 &  0        &  100\%   	\\ 
		trap					&			27196 &  0        &  100\%   	\\ 
		trap\_type				&			27196 &  0        &  100\%   	\\ 
		test\_date				&			27196 &  0        &  100\%   	\\ 
		number\_of\_mosquitoes	&			27196 &  0        &  100\%   	\\ 
		result					&			27196 &  0        &  100\%   	\\ 
		species					&			27196 &  0        &  100\%   	\\ 
		latitude				&			27196 &  4108     &  84,89\%   	\\  
		longitude				&			27196 &  4108     &  84,89\%   	\\  
		\midrule
		Tuple 		&			27196  & 4108	  & 84,89\% 	\\
		Dataset  	&	   		326352 & 8216 	  & 97,48\% \\
		\bottomrule
	\end{tabular}
	\captionof{table}{Analisi di completezza sul dataset \textbf{WNV Mosquito}}
	\label{tab:completezza wnv}
\end{figure}

\subsubsection*{Weather}
\addcontentsline{toc}{subsubsection}{Weather}
Il dataset \textsc{Weather} presenta una completezza del 86,09\%. In particolare 26 attributi su 50 risultano completi, mentre tutti gli altri presentano dei valori mancanti.\\
In tabella \ref{tab:completezza weather} sono presentate le analisi di completezza relative a ciascun attributo e la percentuale della completezza di tupla e di schema.

	\begin{longtable}{lccc}
		\toprule
		\textbf{Attributo} \quad & \textbf{Istanze} & \textbf{Valori nulli} & 
		\textbf{Completezza} \\
		\midrule
		\endfirsthead
		\multicolumn{4}{l}{\footnotesize\itshape Continua dalla pagina precedente} \\
		\toprule
		\textbf{Attributo} \quad & \textbf{Istanze} & \textbf{Valori nulli} & 
		\textbf{Completezza} \\
		\midrule			
		\endhead
		\multicolumn{4}{l}{\footnotesize\itshape Continua nella pagina 
		seguente} \\
		\endfoot
		\endlastfoot
		wban				& 13400 & 0		 	 & 100,00\%  	\\	
		year\_month\_day	& 13400 & 0		 	 & 100,00\%  	\\	
		t\_max				& 13400 & 1723	     & 87,14\%  	\\
		t\_max\_flag		& 13400 & 0		     & 100,00\% 	\\	
		t\_min				& 13400 & 1723	     & 87,14\%  	\\
		t\_min\_flag		& 13400 & 0		     & 100,00\% 	\\	
		t\_avg				& 13400 & 1807	     & 86,51\%  	\\
		t\_avg\_flag		& 13400 & 0		     & 100,00\% 	\\	
		depart				& 13400 & 9561	     & 28,65\%  	\\
		depart\_flag		& 13400 & 0		     & 100,00\% 	\\	
		dew\_point			& 13400 & 61	   	 & 99,54\%  	\\
		dew\_point\_flag	& 13400 & 0		     & 100,00\% 	\\		
		wet\_bulb			& 13400 & 172	     & 98,72\%  	\\
		wet\_bulb\_flag		& 13400 & 0		     & 100,00\% 	\\	
		heat				& 13400 & 1807	     & 86,51\%  	\\
		heat\_flag			& 13400 & 0		     & 100,00\% 	\\	
		cool				& 13400 & 1807	     & 86,51\%  	\\
		cool\_flag			& 13400 & 0		     & 100,00\% 	\\	
		sunrise				& 13400 & 9533	     & 28,86\%  	\\
		sunrise\_flag		& 13400 & 0		     & 100,00\% 	\\	
		sunset				& 13400 & 9533	     & 28,86\%  	\\
		sunset\_flag		& 13400 & 0		     & 100,00\% 	\\	
		code\_sum			& 13400 & 6250	     & 53,36\%  	\\
		code\_sum\_flag		& 13400 & 0		     & 100,00\% 	\\	
		depth				& 13400 & 10922		 & 18,49\%  	\\
		depth\_flag			& 13400 & 0		   	 & 100,00\%		\\	
		water1				& 13400 & 13400		 & 0,00\%		\\
		water1\_flag		& 13400 & 0		   	 & 100,00\%		\\	
		snow\_fall			& 13400 & 10912		 & 18,57\%		\\	
		snow\_fall\_flag	& 13400 & 0		   	 & 100,00\%		\\		
		precip\_total		& 13400 & 1745		 & 86,98\%		\\	
		precip\_total\_flag	& 13400 & 0		   	 & 100,00\%		\\		
		stn\_pressure		& 13400 & 1749		 & 86,95\%		\\	
		stn\_pressure\_flag	& 13400 & 0		   	 & 100,00\%		\\		
		sea\_level			& 13400 & 1771		 & 86,78\%		\\	
		sea\_level\_flag	& 13400 & 0		   	 & 100,00\%		\\		
		result\_speed		& 13400 & 41		 & 99,69\%		\\	
		result\_speed\_flag	& 13400 & 0		   	 & 100,00\%		\\		
		result\_dir			& 13400 & 41		 & 99,69\%		\\			
		result\_dir\_flag	& 13400 & 0		   	 & 100,00\%		\\		
		avg\_speed			& 13400 & 1736		 & 87,04\%		\\	
		avg\_speed\_flag	& 13400 & 0		   	 & 100,00\%		\\		
		max5\_speed			& 13400 & 1728		 & 87,10\%		\\	
		max5\_speed\_flag	& 13400 & 0		   	 & 100,00\%		\\		
		max5\_dir			& 13400 & 1728		 & 87,10\%		\\
		max5\_dir\_flag		& 13400 & 0		   	 & 100,00\%		\\	
		max2\_speed			& 13400 & 1719		 & 87,17\%		\\	
		max2\_speed\_flag	& 13400 & 0		   	 & 100,00\%		\\		
		max2\_dir			& 13400 & 1719		 & 87,17\%		\\
		max2\_dir\_flag		& 13400 & 0		   	 & 100,00\%		\\	
		\midrule
		Tuple 			& 13400  &	13400	 & 0,00\%		\\
		Dataset  		& 670000 &	93188 	 & 86,09\%		\\
		\bottomrule
		
	\end{longtable}
	\captionof{table}{Analisi di completezza sul dataset \textbf{Weather}}
	\label{tab:completezza weather}

\subsubsection*{Stations}
\addcontentsline{toc}{subsubsection}{Stations}
Il dataset \textsc{Stations} presenta una completezza del 82,67\%. Esso risulta essere il meno completo tra i dataset presentati. In particolare quattro attributi su 15 presentano dei valori mancanti.\\
In tabella \ref{tab:completezza stations} sono presentate le analisi di completezza relative a ciascun attributo e la percentuale della completezza di tupla e di schema.

\begin{figure}[H]
	\centering
	\begin{tabular}{lcccc}
		\toprule
		\textbf{Attributo} \quad & \textbf{Istanze} & \textbf{Valori nulli} & 
		\textbf{Completezza} \\
		\midrule
		wban								& 5 & 0	 	& 100,00\%  	\\	
		wmo									& 5 & 3	 	& 40,00\%  		\\	
		callsign							& 5 & 0     & 100,00\%  	\\
		climate\_division\_code				& 5 & 4     & 20,00\% 		\\	
		climate\_division\_state\_code		& 5 & 0     & 100,00\%  	\\
		climate\_division\_station\_code	& 5 & 4     & 20,00\% 		\\	
		name								& 5 & 0     & 100,00\%  	\\
		state								& 5 & 0     & 100,00\% 		\\	
		location							& 5 & 0     & 100,00\%  	\\
		latitude							& 5 & 0     & 100,00\% 		\\	
		longitude							& 5 & 0   	& 100,00\%  	\\
		ground\_height						& 5 & 0     & 100,00\% 		\\		
		station\_height						& 5 & 0     & 100,00\%  	\\
		barometer							& 5 & 2     & 60,00\% 		\\	
		time\_zone							& 5 & 0     & 100,00\%  	\\
		\midrule
		Tuple 						& 5  &	4   & 20,00\%		\\
		Dataset  					& 75 &	13  & 82,67\%		\\
		\bottomrule
	\end{tabular}
	\captionof{table}{Analisi di completezza sul dataset \textbf{Stations}}
	\label{tab:completezza stations}
\end{figure}

\subsection{Miglioramento}

Per il miglioramento della completezza verrà effettuato un completamento di dati incompleti con tecniche specifiche che sfruttano la conoscenza dei dati.\\
\textbf{FIXME questo non l'abbiamo fatto} In particolare, alcuni casi vengono 
facilmente risolti derivando il valore di un specifico attributo da un set di 
altri attributi, per esempio nel dataset \textbf{Weather} l'attributo 
\textsc{t\_avg} può essere facilmente calcolato come la media degli attributi 
\textsc{t\_min} e \textsc{t\_max}.


\subsubsection{Acquisizione di nuovi dati}
Per quando riguarda il dataset \textsc{WNV Mosquito} è stato effettuato un 
completamento dei valori \textit{latitude} e \textit{longitude} mancanti, 
utilizzando il servizio di \textbf{Geocoding} offerto da Google.

In particolare, dagli indirizzi presenti nella colonna \textsc{block} abbiamo 
estratto le corrispondenti coppie di latitudine e longitudine. 
Per verificare l'accuratezza delle posizioni ottenute è stata calcolata la 
distanza geografica tra le coordinate presenti e quelle ottenute dal servizio. 
In questo modo abbiamo potuto verificare che il 95\% delle istanze si trova ad 
una distanza inferiore a 925 metri.

Essendo queste posizioni utilizzate principalmente per individuare le stazioni 
meteorologiche più vicine ad una data trappola, abbiamo ritenuto sufficiente 
l'accuratezza dei risultati del servizio di Geocoding. Abbiamo quindi sfruttato 
le coordinate ottenute per popolare i campi mancanti nel dataset originale, 
raggiungendo una completezza di schema, tupla e attributi pari al 100,00\%.
