La prima dimensione di qualità analizzata è la completezza, ovvero la copertura con cui vengono rappresentati gli insiemi di dati.
In particolare sono state distinte la completezza di tupla, che specifica la presenza di valori nulli in una tupla, la completezza degli attributi ovvero la mancanza di valori in una colonna di attributo e infine la completezza della tabella, ovvero il numero totale di valori non nulli in una tabella.\\

Le metriche utilizzate per la misurazione della completezza sono rispettivamente:
\begin{itemize}
		\item[-] Completezza di tupla: il rapporto tra il numero di valori nulli nella tupla e la cardinalità della tupla (ovvero il numero di attributi);
		\item[-] Completezza di attributo: il rapporto tra il numero di valori nulli della colonna e la cardinalità della colonna (ovvero il numero di tuple);
		\item[-] Completezza di tabella: il rapporto tra il numero di valori nulli nella tabella e la cardinalità della tabella (data dal prodotto cartesiano tra righe e colonne).
\end{itemize}

% Motivo della scelta della dimensione di qualità
% Con cosa è stato effettuato il conteggio dei valori null

La completezza totale dei quattro dataset è del 71,43\%, mentre la completezza delle tuple è solo del 34,96\%.

\begin{figure}[H]
	\centering
	\begin{tabular}{lcccc}
		\toprule
		\textbf{Dataset} \quad & \textbf{Completezza di tupla} & \textbf{Completezza dataset} \\
		\midrule
		WNV Mosquito &		84,89\%  	& 97,48\%  \\ 
		Weather 	 &		 0,00\% 	& 86,09\%  \\ 
		Stations 	 &		20,00\% 	& 82,67\%  \\ 
		\midrule
		Totale 		 &	    34,96\%     & 88,75\%  \\
		\bottomrule
	\end{tabular}
	\captionof{table}{Analisi di completezza sui dataset}
	\label{tab:completezza totale}
\end{figure}

Vengono in seguito mostrate le analisi relative ai dati di ogni dataset.

\subsubsection*{WNV Mosquito}
\addcontentsline{toc}{subsubsection}{WNV Mosquito}
Il dataset \textsc{WNV Mosquito} presenta una completezza del 97,48\%. Esso risulta essere il più completo tra i dataset presentati. In particolare solo due attributi su dodici, \textit{latitudine} e \textit{longitudine}, presentano dei valori mancanti.\\
In tabella \ref{tab:completezza wnv} sono presentate le analisi di completezza relative a ciascun attributo e la percentuale della completezza di tupla e di schema.

\begin{figure}[H]
	\centering
	\begin{tabular}{lcccc}
		\toprule
		\textbf{Attributo} \quad & \textbf{Numero di istanze} & \textbf{Numero di null} & \textbf{Completezza} \\
		\midrule
		SeasonYear &			27196 &  0        &  100\%   \\ 
		Week &					27196 &  0        &  100\%   \\ 
		TestId &				27196 &  0        &  100\%   \\ 
		Block &					27196 &  0        &  100\%   \\ 
		Trap &					27196 &  0        &  100\%   \\ 
		TrapType &				27196 &  0        &  100\%   \\ 
		TestDate &				27196 &  0        &  100\%   \\ 
		NumberOfMosquitoes &	27196 &  0        &  100\%   \\ 
		Result &				27196 &  0        &  100\%   \\ 
		Species &				27196 &  0        &  100\%   \\ 
		Latitude &				27196 &  4108     &  84,89\%   \\  
		Longitude &				27196 &  4108     &  84,89\%   \\  
		\midrule
		Tuple 		&			27196  & 4108	  & 84,89\% 	\\
		Dataset  	&	   		326352 & 8216 	  & 97,48\% \\
		\bottomrule
	\end{tabular}
	\captionof{table}{Analisi di completezza sul dataset \textbf{WNV Mosquito}}
	\label{tab:completezza wnv}
\end{figure}

\subsubsection*{Weather}
\addcontentsline{toc}{subsubsection}{Weather}
Il dataset \textsc{Weather} presenta una completezza del 86,09\%. In particolare 26 attributi su 50 risultano completi, mentre tutti gli altri presentano dei valori mancanti.\\
In tabella \ref{tab:completezza weather} sono presentate le analisi di completezza relative a ciascun attributo e la percentuale della completezza di tupla e di schema.

	\begin{longtable}{lcccc}
		\toprule
		\textbf{Attributo} \quad & \textbf{Numero di istanze} & \textbf{Numero di null} & \textbf{Completezza} \\
		\midrule
		\endfirsthead
		\multicolumn{5}{l}{\footnotesize\itshape Continua dalla pagina precedente} \\
		\toprule
		\textbf{Attributo} \quad & \textbf{Numero di istanze} & \textbf{Numero di null} & \textbf{Completezza} \\
		\midrule			
		\endhead
		\multicolumn{5}{l}{\footnotesize\itshape Continua nella prossima pagina} \\
		\endfoot
		\endlastfoot
		WBAN			& 13400 & 0		 	 & 100,00\%  	\\	
		YearMonthDay	& 13400 & 0		 	 & 100,00\%  	\\	
		Tmax			& 13400 & 1723	     & 87,14\%  	\\
		TmaxFlag		& 13400 & 0		     & 100,00\% 	\\	
		Tmin			& 13400 & 1723	     & 87,14\%  	\\
		TminFlag		& 13400 & 0		     & 100,00\% 	\\	
		Tavg			& 13400 & 1807	     & 86,51\%  	\\
		TavgFlag		& 13400 & 0		     & 100,00\% 	\\	
		Depart			& 13400 & 9561	     & 28,65\%  	\\
		DepartFlag		& 13400 & 0		     & 100,00\% 	\\	
		DewPoint		& 13400 & 61	   	 & 99,54\%  	\\
		DewPointFlag	& 13400 & 0		     & 100,00\% 	\\		
		WetBulb			& 13400 & 172	     & 98,72\%  	\\
		WetBulbFlag		& 13400 & 0		     & 100,00\% 	\\	
		Heat			& 13400 & 1807	     & 86,51\%  	\\
		HeatFlag		& 13400 & 0		     & 100,00\% 	\\	
		Cool			& 13400 & 1807	     & 86,51\%  	\\
		CoolFlag		& 13400 & 0		     & 100,00\% 	\\	
		Sunrise			& 13400 & 9533	     & 28,86\%  	\\
		SunriseFlag		& 13400 & 0		     & 100,00\% 	\\	
		Sunset			& 13400 & 9533	     & 28,86\%  	\\
		SunsetFlag		& 13400 & 0		     & 100,00\% 	\\	
		CodeSum			& 13400 & 6250	     & 53,36\%  	\\
		CodeSumFlag		& 13400 & 0		     & 100,00\% 	\\	
		Depth			& 13400 & 10922		 & 18,49\%  	\\
		DepthFlag		& 13400 & 0		   	 & 100,00\%		\\	
		Water1			& 13400 & 13400		 & 0,00\%		\\
		Water1Flag		& 13400 & 0		   	 & 100,00\%		\\	
		SnowFall		& 13400 & 10912		 & 18,57\%		\\	
		SnowFallFlag	& 13400 & 0		   	 & 100,00\%		\\		
		PrecipTotal		& 13400 & 1745		 & 86,98\%		\\	
		PrecipTotalFlag	& 13400 & 0		   	 & 100,00\%		\\		
		StnPressure		& 13400 & 1749		 & 86,95\%		\\	
		StnPressureFlag	& 13400 & 0		   	 & 100,00\%		\\		
		SeaLevel		& 13400 & 1771		 & 86,78\%		\\	
		SeaLevelFlag	& 13400 & 0		   	 & 100,00\%		\\		
		ResultSpeed		& 13400 & 41		 & 99,69\%		\\	
		ResultSpeedFlag	& 13400 & 0		   	 & 100,00\%		\\		
		ResultDir		& 13400 & 41		 & 99,69\%		\\			
		ResultDirFlag	& 13400 & 0		   	 & 100,00\%		\\		
		AvgSpeed		& 13400 & 1736		 & 87,04\%		\\	
		AvgSpeedFlag	& 13400 & 0		   	 & 100,00\%		\\		
		Max5Speed		& 13400 & 1728		 & 87,10\%		\\	
		Max5SpeedFlag	& 13400 & 0		   	 & 100,00\%		\\		
		Max5Dir			& 13400 & 1728		 & 87,10\%		\\
		Max5DirFlag		& 13400 & 0		   	 & 100,00\%		\\	
		Max2Speed		& 13400 & 1719		 & 87,17\%		\\	
		Max2SpeedFlag	& 13400 & 0		   	 & 100,00\%		\\		
		Max2Dir			& 13400 & 1719		 & 87,17\%		\\
		Max2DirFlag		& 13400 & 0		   	 & 100,00\%		\\	
		\midrule
		Tuple 			& 13400  &	13400	 & 0,00\%		\\
		Dataset  		& 670000 &	93188 	 & 86,09\%		\\
		\bottomrule
		
	\end{longtable}
	\captionof{table}{Analisi di completezza sul dataset \textbf{Weather}}
	\label{tab:completezza weather}

\subsubsection*{Stations}
\addcontentsline{toc}{subsubsection}{Stations}
Il dataset \textsc{Stations} presenta una completezza del 82,67\%. Esso risulta essere il meno completo tra i dataset presentati. In particolare quattro attributi su 15 presentano dei valori mancanti.\\
In tabella \ref{tab:completezza stations} sono presentate le analisi di completezza relative a ciascun attributo e la percentuale della completezza di tupla e di schema.

\begin{figure}[H]
	\centering
	\begin{tabular}{lcccc}
		\toprule
		\textbf{Attributo} \quad & \textbf{Numero di istanze} & \textbf{Numero di null} & \textbf{Completezza} \\
		\midrule
		WBAN						& 5 & 0	 	& 100,00\%  	\\	
		WMO							& 5 & 3	 	& 40,00\%  		\\	
		CallSign					& 5 & 0     & 100,00\%  	\\
		ClimateDivisionCode			& 5 & 4     & 20,00\% 		\\	
		ClimateDivisionStateCode	& 5 & 0     & 100,00\%  	\\
		ClimateDivisionStationCode	& 5 & 4     & 20,00\% 		\\	
		Name						& 5 & 0     & 100,00\%  	\\
		State						& 5 & 0     & 100,00\% 		\\	
		Location					& 5 & 0     & 100,00\%  	\\
		Latitude					& 5 & 0     & 100,00\% 		\\	
		Longitude					& 5 & 0   	& 100,00\%  	\\
		GroundHeight				& 5 & 0     & 100,00\% 		\\		
		StationHeight				& 5 & 0     & 100,00\%  	\\
		Barometer					& 5 & 2     & 60,00\% 		\\	
		TimeZone					& 5 & 0     & 100,00\%  	\\
		\midrule
		Tuple 						& 5  &	4   & 20,00\%		\\
		Dataset  					& 75 &	13  & 82,67\%		\\
		\bottomrule
	\end{tabular}
	\captionof{table}{Analisi di completezza sul dataset \textbf{Stations}}
	\label{tab:completezza stations}
\end{figure}

\subsection{Miglioramento}

Per il miglioramento della completezza verrà effettuato un completamento di dati incompleti con tecniche specifiche che sfruttano la conoscenza dei dati.\\
In particolare, alcuni casi vengono facilmente risolti derivando il valore di un specifico attributo da un set di altri attributi, per esempio nel dataset \textbf{Weather} l'attributo \textsc{Tavg} può essere facilmente calcolato come la media degli attributi \textsc{Tmin} e \textsc{Tmax}.


\subsubsection{Acquisizione di nuovi dati}
Per quando riguarda il dataset \textsc{WNV Mosquito} è stata effettuato un completamento delle istanze mancanti degli attributi \textit{latitudine} e \textit{longitudine} utilizzando il servizio \textbf{Geocoding} Google.

In particolare utilizzando gli indirizzi presenti nella colonna \textsc{Block} abbiamo estratto le corrispondenti coppie di coordinate geografiche latitudine e longitudine. 
Per verificarne l'accuratezza è stato effettuato il calcolo della distanza geografica tra le istanze non nulle e quelle ottenute dal servizio. In particolare abbiamo potuto verificare che il 95\% delle coppie di istanze si trovano ad una distanza inferiore a 925m.

Dunque, essendo le posizioni degli indirizzi utilizzate per cercare la stazione di rilevamento meteo più vicine, ed essendo l'accuratezza sufficientemente alta per questo scopo, abbiamo deciso di utilizzare le coordinate calcolate con il servizio di Geocoding per riempire i campi mancanti del dataset, ottenendo una completezza totale di schema, tupla e attributi pari al 100,00\%.
