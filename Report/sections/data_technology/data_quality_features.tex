\section{Completezza}
% FIXME espandere
In seguito alla selezione delle 66 features, la completezza complessiva del 
dataset risulta essere del 100\%, così come quella di attributo e di tupla.

\section{Leggibilità}
% FIXME rivedere
Dopo le modifiche apportate al dataset, in particolare quelle per convertire le 
unità di misura dal sistema anglosassone a quello internazionale e ... la 
leggibilità risulta notevolmente migliorata.

Il dataset finale contiene per lo più attributi di tipo numerico, che esprimono 
valori di grandezze fisiche (distanze, temperature, velocità), che sono 
autoesplicativi. Potrebbe essere utile invece rendere immediata all'utente la 
visualizzazione del risultato delle analisi, del giorno in cui sono state 
effettuate e soprattutto le condizioni meteo poiché ora vengono utilizzati dei 
codici numerici. 

A causa di ciò è stato pensato di utilizzare una vista SQL per 
evidenziare questi dati. Una query sugli attributi discussi in precedenza 
produce i risultati visibili in Tabella~\ref{tab:dataset-before-view}, 
mentre applicando la query SQL mostrata nel Listing~\ref{code:sql-view} 
vengono restituiti i record mostrati nella 
Tabella~\ref{tab:dataset-after-view}.


\begin{figure}[H]
	\centering
	\begin{tabular}{lcccc}
		\toprule
		test\_id & result & day\_of\_week & cod\_sum\_ra & cod\_sum\_ts \\
		\midrule
		23505 & 0 & 2 & 3 & 0 \\
		23109 &	0 & 2 & 0 & 0 \\
		45618 &	1 & 3 & 1 & 2 \\
		40026 &	1 & 4 & 1 & 0 \\
		\bottomrule
	\end{tabular}
	\captionof{table}{Risultato query dataset senza view}
	\label{tab:dataset-before-view}
\end{figure}

\begin{lstlisting}[
label={code:sql-view},
caption={Query SQL per la leggibilità degli attributi},
captionpos=b,
breaklines=true,                                    
language=SQL,
frame=ltrb,
framesep=5pt,
basicstyle=\normalsize,
keywordstyle=\ttfamily\color{OliveGreen},
identifierstyle=\ttfamily\color{MidnightBlue}\bfseries,
commentstyle=\color{Brown},
stringstyle=\ttfamily,
showstringspaces=false
]
SELECT test_id, 
result, 
day_of_week, 
weather_conditions
FROM "ReadableFusedDataset";
\end{lstlisting}

In particolare è possibile notare come gli attributi \texttt{code\_sum\_ra} 
e \texttt{code\_sum\_ts} sono stati collassati in un unico attributo 
\texttt{weather\_conditions} visibile nella 
Tabella~\ref{tab:dataset-after-view} che racchiude la rappresentazione 
testuale di tutti i fenomeni meteorologici registrati durante il test.

\begin{figure}[H]
	\centering
	\begin{tabular}{lccc}
		\toprule
		test\_id & result & day\_of\_week & weather\_conditions \\
		\midrule
		23505 & Negativo & Mercoledì & {Pioggia} \\
		23109 &	Negativo & Mercoledì & {} \\
		45618 &	Positivo & Giovedì & {Pioggia,Temporale} \\
		40026 &	Positivo & Venerdì & {Pioggia} \\
		\bottomrule
	\end{tabular}
	\captionof{table}{Risultato query dataset attraverso la view}
	\label{tab:dataset-after-view}
\end{figure}
