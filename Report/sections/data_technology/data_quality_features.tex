\section{Completezza}
In seguito alla selezione delle 66 feature finali, la completezza complessiva 
del dataset risulta essere del 100\%, così come quella di attributo e di tupla.

\section{Leggibilità}
Grazie alle modifiche apportate al dataset, in particolare la conversione delle 
unità di misura, la leggibilità risulta notevolmente migliorata.

Il dataset finale contiene soprattutto attributi di tipo numerico che esprimono 
valori di grandezze fisiche (distanze, temperature, velocità, \dots), per lo 
più autoesplicativi.

Potrebbe essere utile invece migliorare la visualizzazione del risultato delle 
analisi, del giorno in cui sono state effettuate e delle condizioni meteo. Per 
ottenere una rappresentazione più intuitiva di questi attributi abbiamo creato 
una vista SQL chiamata \textsc{ReadableFusedDataset}.

Una query sulle colonne elencate produce i risultati visibili in 
Tabella~\ref{tab:dataset-before-view}, mentre eseguendo la query SQL mostrata 
nel Listing~\ref{code:sql-view} vengono restituiti i record mostrati nella 
Tabella~\ref{tab:dataset-after-view}.

\begin{figure}[H]
	\centering
	\begin{tabular}{lcccc}
		\toprule
		test\_id & result & day\_of\_week & cod\_sum\_ra & cod\_sum\_ts \\
		\midrule
		23505 & 0 & 2 & 3 & 0 \\
		23109 &	0 & 2 & 0 & 0 \\
		45618 &	1 & 3 & 1 & 2 \\
		40026 &	1 & 4 & 1 & 0 \\
		\bottomrule
	\end{tabular}
	\captionof{table}{Risultato della query sul dataset senza view}
	\label{tab:dataset-before-view}
\end{figure}

\begin{lstlisting}[
label={code:sql-view},
caption={Query SQL per la leggibilità degli attributi},
captionpos=b,
breaklines=true,                                    
language=SQL,
frame=ltrb,
framesep=5pt,
basicstyle=\normalsize,
keywordstyle=\ttfamily\color{OliveGreen},
identifierstyle=\ttfamily\color{MidnightBlue}\bfseries,
commentstyle=\color{Brown},
stringstyle=\ttfamily,
showstringspaces=false
]
SELECT test_id, 
	result, 
	day_of_week, 
	weather_conditions
FROM "ReadableFusedDataset";
\end{lstlisting}

In particolare è possibile notare come gli attributi \texttt{code\_sum\_*} 
siano riuniti in un unico attributo \texttt{weather\_conditions} nella 
Tabella~\ref{tab:dataset-after-view}, che racchiude la rappresentazione 
testuale di tutti i fenomeni meteorologici registrati durante il test.

\begin{figure}[H]
	\centering
	\begin{tabular}{lccc}
		\toprule
		test\_id & result & day\_of\_week & weather\_conditions \\
		\midrule
		23505 & Negativo & Mercoledì & {Pioggia} \\
		23109 &	Negativo & Mercoledì & {} \\
		45618 &	Positivo & Giovedì & {Pioggia,Temporale} \\
		40026 &	Positivo & Venerdì & {Pioggia} \\
		\bottomrule
	\end{tabular}
	\captionof{table}{Risultato query dataset attraverso la view}
	\label{tab:dataset-after-view}
 \end{figure}

\section{Accuratezza}
È stata anche eseguita un'analisi di accuratezza delle feature per le quali è 
stato possibile individuare un dominio ben definito e nella Tabella 
\ref{tab:features-accuracy} sono mostrati i risultati ottenuti. Come è 
possibile notare, tutte le feature analizzate rispettano il dominio di 
riferimento. 

\vspace{.5em}
\begin{centering}
	\begin{tabular}{lccc}
		\toprule
		feature & dominio & \% valori esterni al dominio \\
		\midrule
		season\_year & $[2007, 2017]$ & $0$ \\
		week &	$[1, 53]$ & $0$ \\
		number\_of\_mosquitoes & $[0, 50]$ & 0 \\
		day\_of\_week &	$[0, 6]$ & $0$ \\
		precip\_total &	$\geq 0$ & $0$ \\
		stn\_pressure &	$\geq 0$ & $0$ \\
		sea\_level &	$\geq 0$ & $0$ \\
		result\_speed &	$\geq 0$ & $0$ \\
		result\_dir &	$[0, 360]$ & $0$ \\
		avg\_speed &	$\geq 0$ & $0$ \\
		max5\_speed &	$\geq 0$ & $0$ \\
		max5\_dir &	$[0, 360]$ & $0$ \\
		max2\_speed &	$\geq 0$ & $0$ \\
		max2\_dir &	$[0, 360]$ & $0$ \\
		station\_latitude & $[-90, 90]$ & $0$ \\
		station\_longitude & $[-180, 180]$ & $0$ \\
		station\_ground\_height &	$\geq 0$ & $0$ \\
		station\_height & $\geq 0$ & $0$ \\
		avg\_speed & $\geq 0$ & $0$ \\
		block\_latitude & $[-90, 90]$ & $0$ \\
		block\_longitude & $[-180, 180]$ & $0$ \\
		\bottomrule
	\end{tabular}
	\captionof{table}{Analisi di accuratezza delle feature rispetto al dominio 
		di riferimento}
	\label{tab:features-accuracy}
\end{centering}
