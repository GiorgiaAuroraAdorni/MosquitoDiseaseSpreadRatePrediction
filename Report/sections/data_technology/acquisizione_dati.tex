Nel dataset \textsc{WNV Mosquito} abbiamo voluto aggiungere l'attributo 
\texttt{day\_of\_week} che corrisponde al giorno della settimana in cui è stato 
effettuato il test. questo è stato utile anche per derivare delle statistiche, 
come per esempio il giorno della settimana in cui è stato effettuato il maggior 
numero di test. 
\\

Un analisi del dataset \textsc{Weather} ha mostrato come alcuni codici della 
colonna \texttt{code\_sum} comparissero molto raramente. Per rendere questi 
valori più adatti all'utilizzo come feature, abbiamo deciso di aggregarli 
manualmente in categorie più ampie, in base alla somiglianza dei fenomeni 
meteorologici rappresentati:
\begin{itemize}
	\item I codici \texttt{GR} (grandine, 1 occorrenza), \texttt{GS} (piccola 
	grandine, 3 occorrenze) e \texttt{PL} (pioggia gelata, 33 occorrenze) sono 
	stati trasformati nel codice aggregato \texttt{HAIL} (grandine, 37 
	occorrenze).
	
	\item I codici \texttt{SQ} (raffiche di vento, 8 occorrenze), \texttt{FC} 
	(nubi a imbuto, 1 occorrenza) sono stati trasformati nel codice aggregato 
	\texttt{WIND} (forte vento, 9 occorrenze).
	
	\item I codici \texttt{FU} (fumo, 19 occorrenze), \texttt{DU} (polvere, 1 
	occorrenza) e sono stati trasformati nel codice aggregato \texttt{SMOKE} 
	(fumo e polvere, 20 occorrenze).
	
	\item Tutti i rimanenti codici compaiono più di 110 volte nell'arco 
	temporale analizzato e sono stati mantenuti invariati.
\end{itemize}

Nello stesso dataset, a seguito dell'aggregazione delle osservazioni 
giornaliere in settimanali, è stata aggiunta la colonna 
\texttt{days\_per\_week}, rappresentante il numero di giorni settimanali 
aggregati. Nella maggior parte dei casi questo valore corrisponde a 7, ovvero 
l'aggregazione di tutti i giorni della settimana.

\textbf{FIXME} aggiungere spiegazione sulla trasformazione di weather da daily 
a weakly
