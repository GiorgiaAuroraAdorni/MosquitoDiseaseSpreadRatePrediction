Per migliorare la completezza, nel dataset \textsc{WNV Mosquito} è stato 
possibile effettuare un completamento dei valori \texttt{latitude} e 
\texttt{longitude} mancanti, utilizzando il servizio di \textbf{Geocoding} 
offerto da Google.

In particolare, dagli indirizzi presenti nella colonna \texttt{block} abbiamo 
estratto le corrispondenti coppie di latitudine e longitudine. 
Per verificare l'accuratezza delle posizioni ottenute è stata calcolata la 
distanza geografica tra le coordinate presenti e quelle ottenute dal servizio. 
In questo modo abbiamo potuto verificare che il 95\% delle istanze si trova ad 
una distanza inferiore a 925 metri.

Essendo queste posizioni utilizzate principalmente per individuare le stazioni 
meteorologiche più vicine ad una data trappola, abbiamo ritenuto sufficiente 
l'accuratezza dei risultati del servizio di Geocoding. Quindi, sfruttando 
le coordinate ottenute, sono stati popolati i campi mancanti nel dataset 
originale, raggiungendo una completezza di schema, tupla e attributi pari al 
100,00\%.
\\
Sempre in questo dataset abbiamo voluto aggiungere l'attributo 
\texttt{day\_of\_week} che corrisponde al giorno della settimana in cui è stato 
effettuato il test. questo è stato utile anche per derivare delle statistiche, 
come per esempio il giorno della settimana in cui è stato effettuato il maggior 
numero di test. 
\\

Nel dataset \textsc{Weather}, a seguito dell'aggregazione delle osservazioni 
giornaliere in settimanali, è stata aggiunta la colonna 
\texttt{days\_per\_week}, rappresentante il numero di giorni settimanali 
aggregati. Nella maggior parte dei casi questo valore corrisponde a 7, ovvero 
l'aggregazione di tutti i giorni della settimana.

%fixme 
% aggiungere parte su code_sum 
