Una volta terminata la valutazione dei dataset tramite le dimensioni qualità, è stata eseguita una deduplicazione dei dataset, ovvero l'identificazione di tutte le coppie o gruppi di tuple corrispondenti allo stesso oggetto del mondo reale.

La deduplicazione è stata eseguita utilizzando il software Power BI in particolare ricorrendo allo strumento Power Query per l'inserimento, la trasformazione, l'integrazione e l'arricchimento dei dati.
\\\\	
È stata effettuata la deduplicazione del dataset \textbf{WNV Mosquito}, utilizzando analizzando l'attributo \textsc{Block}. In particolare è stata riscontrata la presenza di alcune tuple duplicate, che differivano per i campi \textsc{Latitude} e \textsc{Longitude}. Calcolando la distanza tra le diverse coordinate abbiamo verificato che le differenze erano solo di alcuni metri.
Dunque è stato deciso di tenere randomicamente una delle tuple duplicate data la scarsa differenza tra le due e la poca importanza della precisione, poiché nel nostro modello predittivo le coordinate di ogni indirizzo vengono utilizzate solo per cercare una corrispondenza con la stazione di rilevamento meteo più vicina.
\\\\
Sempre sul dataset \textbf{WNV Mosquito} è stata provata l'esecuzione della deduplicazione questa volta utilizzando come input la colonna \textsc{Trap} e i campi \textsc{Latitude}, \textsc{Longitude} e \textsc{Block}. In questo caso viene erroneamente rilevata la presenza di due tuple duplicate, in cui la stessa trappola è situata in due posti diversi. 

Discriminando attraverso l'attributo \textsc{Season Year} non vengono rilevate più le tuple duplicate. Questo deriva dal fatto che la trappola potrebbe essere stata spostata negli anni.

Lo stesso problema si verifica nel caso vengano presi in input la colonna \textsc{Trap} e il campo \textsc{Trap Type}. Questa volta sono 17 le tuple rilevate come duplicate, ma ancora una volta è possibile discriminare attraverso l'attributo \textsc{Season Year}.
