In seguito alla valutazione dei dataset tramite le dimensioni di qualità è 
stata eseguita l'operazione di deduplicazione sui singoli dataset, ovvero 
l'identificazione di quelle coppie o gruppi di tuple corrispondenti ad uno 
stesso oggetto del mondo reale.

È stato utilizzato il software Power BI di Microsoft, 
in particolare ricorrendo allo strumento Power Query, per l'inserimento, la 
trasformazione, l'integrazione e l'arricchimento dei dati.

È stata effettuata la deduplicazione del dataset \textbf{WNV Mosquito}, 
analizzando l'attributo \textsc{block}. In particolare è stata riscontrata 
la presenza di alcune tuple duplicate, che differivano per i campi 
\textsc{latitude} e \textsc{longitude}. Calcolando la distanza tra le diverse 
coordinate abbiamo verificato che le differenze erano solo di alcuni metri.
Dunque è stato deciso di tenere \textbf{FIXME randomicamente} una delle tuple 
duplicate data la scarsa differenza tra le due e la poca importanza della 
precisione, poiché nel nostro modello predittivo le coordinate di ogni 
indirizzo vengono utilizzate solo per cercare una corrispondenza con la 
stazione di rilevamento meteo più vicina.

Sempre sul dataset \textbf{WNV Mosquito} è stata analizzata la duplicazione 
delle colonne \textsc{latitude}, \textsc{longitude} e \textsc{block} rispetto a 
\textsc{trap}. In questo caso abbiamo rilevato la presenza di una stessa 
trappola situata in luoghi diversi. Aggiungendo l'attributo 
\textsc{season\_year} non abbiamo più riscontrato tuple duplicate. Abbiamo 
quindi ipotizzato che la stessa trappola fosse stata posizionata diversamente 
negli anni.

Una situazione analoga si verifica analizzando il campo \textsc{trap\_type} 
rispetto a \textsc{trap}. Vengono rilevate 17 tuple duplicate, ma ancora una 
volta è possibile discriminarle attraverso l'attributo \textsc{season\_year}.
