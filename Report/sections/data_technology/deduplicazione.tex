È stata poi eseguita l'operazione di deduplicazione sui singoli dataset, ovvero 
l'identificazione di quelle coppie o gruppi di tuple corrispondenti ad uno 
stesso oggetto del mondo reale.

È stato utilizzato il software Power BI di Microsoft, in particolare ricorrendo 
allo strumento Power Query, per l'inserimento, la trasformazione, 
l'integrazione e l'arricchimento dei dati.

È stata effettuata la deduplicazione del dataset \textsc{Block}, considerando
l'attributo \texttt{block} come chiave primaria. In particolare è stata 
riscontrata la presenza di alcune tuple duplicate, che differivano per i campi 
\texttt{latitude} e \texttt{longitude}. Calcolando la distanza tra le diverse 
coordinate abbiamo verificato che le differenze erano solo di alcuni metri.
Dunque è stato deciso in modo arbitrario di tenere una delle tuple duplicate 
data la scarsa differenza tra le due e la poca precisione richiesta, 
poiché nel nostro modello predittivo le coordinate di ogni indirizzo vengono 
utilizzate solo per cercare una corrispondenza con la stazione di rilevamento 
meteo più vicina.

Sul dataset \textsc{WNV Mosquito} è stata analizzata la duplicazione 
delle colonne \texttt{latitude}, \texttt{longitude} e \texttt{block} rispetto 
alla chiave primaria \texttt{trap}. In questo caso abbiamo rilevato la presenza 
di una stessa trappola situata in luoghi diversi. Aggiungendo l'attributo 
\texttt{season\_year} alla chiave non abbiamo più riscontrato tuple duplicate. 
Abbiamo quindi ipotizzato che la stessa trappola fosse stata posizionata 
diversamente negli anni.

Una situazione analoga si verifica analizzando il campo \texttt{trap\_type} 
rispetto a \texttt{trap}. Vengono rilevate 17 tuple duplicate, ma ancora una 
volta è possibile discriminarle attraverso l'attributo \texttt{season\_year}.
