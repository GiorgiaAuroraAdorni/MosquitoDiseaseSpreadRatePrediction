\setcounter{chapter}{0}
\part{Machine Learning}
\chapter{Creazione dei training set}
\sections{machine_learning/creazione}

\chapter{Analisi esplorativa del training set}
\sections{machine_learning/analisi_esplorativa}

\chapter{Descrizione e motivazione dei modelli di machine learning utilizzati}

\sections{machine_learning/modello_random_forest}

\section{Support Vector Machine}
\sections{machine_learning/svm}

\section{Neural Network}
\sections{machine_learning/nn}

\chapter{Esperimenti}
% FIXME: rivedere se per ogni modello...
Per ogni modello, la creazione di training e validation set è stata realizzata 
utilizzando due tecniche diverse: è stato effettuato un esperimento in cui 
viene utilizzato un \textit{holdout} 80--20 ed uno che ricorre ad una 
\textit{10-fold cross validation} con dataset tutti della stessa dimensione.

\sections{machine_learning/esperimento_random_forest}

\chapter{Analisi dei risultati ottenuti}
\sections{machine_learning/risultati}

\chapter{Conclusioni}
\sections{machine_learning/conclusioni}
