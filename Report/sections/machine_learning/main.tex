\setcounter{chapter}{0}
\part{Machine Learning}
\chapter{Creazione del training set}
\sections{machine_learning/creazione}
Per ogni modello, la creazione dei set di training e validation è stata 
realizzata utilizzando due tecniche diverse: in un primo esperimento è stato 
utilizzato un \textit{holdout} 80--20, in un secondo si è ricorso ad una 
\textit{10-fold cross validation}.

Le feature utilizzate per i modelli sono le seguenti:
\begin{itemize}
	\item \textbf{Random Forest}: è stato deciso di utilizzare tutte quelle 
	disponibili, eliminando solamente le colonne \texttt{test\_date} e 
	\texttt{main\_trap}, poiché contenenti più di 53 categorie;
	\item \textbf{Support Vector Machine}: sono state utilizzate solamente le 
	colonne numeriche, eliminando quelle che assumono sempre lo stesso 
	valore sull'intero dataset;
	\item \textbf{Neural Network}: ancora una vota vengono utilizzate solamente 
	le colonne numeriche, eliminando quelle che mantengono lo stesso valore 
	sull'intero set di dati, e	normalizzando le restanti secondo una 
	distribuzione normale standard.
\end{itemize} 

\chapter{Analisi esplorativa del training set}
\sections{machine_learning/analisi_esplorativa}

\chapter{Modelli utilizzati}

\sections{machine_learning/modello_random_forest}
\sections{machine_learning/modello_svm}
\sections{machine_learning/modello_mlp}

\chapter{Esperimenti}
% FIXME: rivedere se per ogni modello...
Per ogni modello, la creazione di training e validation set è stata realizzata 
utilizzando due tecniche diverse: è stato effettuato un esperimento in cui 
viene utilizzato un \textit{holdout} 80--20 ed uno che ricorre ad una 
\textit{10-fold cross validation} con dataset tutti della stessa dimensione.

\sections{machine_learning/esperimento_random_forest}
\sections{machine_learning/esperimento_svm}
\sections{machine_learning/esperimento_mlp}

\chapter{Analisi dei risultati ottenuti}
\sections{machine_learning/risultati}

\chapter{Conclusioni}
\sections{machine_learning/conclusioni}
