È stato scelto di provare ad utilizzare il modello di classificazione 
Support Vector Machine (SVM) poichè la classificazione di un dataset contenente 
dati reali e comprensivo di circa 70 features potrebbe risultare difficile, a 
causa della bassa possibilità che i dati siano linearmente separabili. 
Con SVM è possibile utilizzare il metodo Kernel tramite delle funzioni 
che permettono di aumentare la dimensionalità dei dati senza dover 
calcolare nuovamente la loro posizione all'interno del feature space; in questo 
modo è possibile trovare un iperpiano di separazione in grado di classificare 
i dati.

La funzione kernel (K) utilizzata è di tipo 
Radial Basis Function (RBF) ed è definita nell'Equazione \ref{eq:rbf}.

\begin{equation}\label{eq:rbf}
    K(x, x') = e^{- \gamma \|x - x'\|^2}
\end{equation}

Il classificatore può essere addestrato utilizzando due parametri per 
controllare il grado di linearità dell'iperpiano ($\gamma$) e il margine ($C$) 
tra i vettori di supporto:

\begin{itemize}
    \item $\gamma$: è il parametro della funzione kernel \ref{eq:rbf}, minore è 
    il valore e maggiormente l'iperpiano assumerà una forma lineare. 
    Aumentando troppo questo valore si rischia l'overfitting del modello.
    \item $C$: è responsabile di controllare l'ampiezza del margine tra i 
    vettori di supporto, maggiore è il suo valore minore sarà l'ampiezza del 
    margine portando a una diminuzione della proprietà di generalizzazione.
\end{itemize}

Attraverso il processo di tuning del modello è stato scelto di utilizzare il 
valore $0.01$ per il parametro $\gamma$ e $10$ per $C$.
