\section{Random Forest}
Il primo modello implementato è stato Random Forest. Questo è stato scelto 
poiché in grado di maneggiare diversi tipi di variabili, perché robusto 
rispetto agli outlier e perché dà una stima dell'errore e dell'importanza delle 
variabili. 

I modelli di tipo Random Forest convergono sempre e non presentano grandi  
problemi di overfitting.

Il modello è di semplice utilizzo, in quanto prevede l’inserimento di soltanto 
due parametri: il numero di variabili nel sottoinsieme di variabili casuali 
considerate in ogni nodo ed il numero di alberi da addestrare nella foresta. 
Inoltre non è particolarmente sensibile ai valori di questi parametri, per 
questo è stata utilizzata la configurazione di default:
\begin{itemize}
	\item \textbf{Number of variables tried at each split:} 7
	\item \textbf{Number of trees:} 500
\end{itemize}
