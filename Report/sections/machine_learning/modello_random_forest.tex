\section{Random Forest}
Il primo modello implementato è Random Forest.

Questo è stato scelto poiché in grado di maneggiare diversi tipi di variabili, 
perché è robusto rispetto agli outlier, da una stima dell'errore e 
dell'importanza delle variabili. 
%FIXME questo non so come esprimerlo, anche perchè ho fatto sia validation che 
%convalida incrociata
I modelli di tipo Random Forest convergono sempre e non presentano problemi di 
overfitting. Per questo motivo Random Forest non necessita della messa in atto 
di tecniche di validazione incrociata o della verifica su un set separato di 
variabili per avere una valutazione imparziale dell'errore, in quanto ciò è 
garantito intrinsecamente dal metodo.

Il modello è di semplice utilizzo, in quanto prevede l’inserimento di soltanto 
due parametri (il numero di variabili nel sottoinsieme di variabili casuali 
usate in ogni nodo ed il numero di alberi nella foresta) e non è molto 
sensibile ai loro valori. Per questo motivo è stata utilizzata la 
configurazione di default.
%(i due parametri dovrebbero essere \texttt{ntree} e \texttt{mtry}:  
%Number of trees: 500
%No. of variables tried at each split: 7
%)
