\section{Descrizione del dominio di riferimento}

Il virus del Nilo occidentale (noto anche come West Nile Virus, WNV) infetta 
ogni anno migliaia di persone, provocando nel 20\% circa dei casi sintomi che 
variano da forti febbri a gravi complicazioni neurologiche fino ad arrivare  
anche alla morte.

Oltre agli esseri umani il virus colpisce anche altri animali, tra cui 
uccelli e cavalli, causando in questi ultimi tassi di mortalità che raggiungono 
il 40\%. Isolato per la prima volta nel 1937 nel distretto di West Nile in 
Uganda, da cui prende il nome, si è oggi diffuso in tutto il mondo.

Per questi motivi, il controllo e la prevenzione delle infezioni da WNV 
risultano essere argomenti di grande interesse. 

Le zanzare infette costituiscono il principale vettore di trasmissione del 
virus agli esseri umani. Quando i primi casi umani furono riportati a Chicago 
nel 2002, il Dipartimento di Sanità Pubblica della città ha avviato un esteso 
programma di sorveglianza e controllo che resta tutt'oggi in vigore.

Dalla fine della primavera all'inizio dell'autunno, numerose trappole per 
zanzare vengono distribuite per la città, andando ogni settimana a testare la 
presenza del virus negli esemplari catturati. 
%I risultati di questi test influenzano quando e dove la città spruzzerà i pesticidi dispersi nell'aria per controllare le popolazioni di zanzare adulte.

\textbf{FIXME: questo potremmo spostarlo nel capitolo Obiettivi no?} Grazie ai 
dati relativi alle condizioni meteorologiche e all'ubicazione delle trappole il 
nostro obiettivo è quello di prevedere se le zanzare risultino positive al 
virus del Nilo occidentale.

%quando e dove diverse specie di zanzare saranno testate positive per il virus del Nilo occidentale. 

%Un metodo più accurato di previsione dei focolai del virus del Nilo occidentale nelle zanzare aiuterà la città di Chicago e il CPHD ad allocare in modo più efficiente ed efficace le risorse per prevenire la trasmissione di questo virus potenzialmente letale.

% L’obiettivo di questo elaborato è ... sfruttando le moderne tecniche di machine learning.


\section{Scelte di design per la creazione del dataset}

\section{Eventuali ipotesi o assunzioni}

