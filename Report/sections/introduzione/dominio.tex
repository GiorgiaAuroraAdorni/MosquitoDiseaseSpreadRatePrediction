\section{Dominio di riferimento}

Il virus del Nilo occidentale (noto anche come West Nile Virus, WNV) infetta 
ogni anno migliaia di persone, provocando nel 20\% circa dei casi sintomi che 
variano da forti febbri a gravi complicazioni neurologiche fino ad arrivare  
anche alla morte.

Oltre agli esseri umani, il virus colpisce anche animali quali uccelli e 
cavalli, causando in questi ultimi tassi di mortalità che raggiungono il 40\%. 
Isolato per la prima volta nel 1937 nel distretto di West Nile in Uganda, da 
cui prende il nome, si è oggi diffuso in tutto il mondo.

Per questi motivi, il controllo e la prevenzione delle infezioni da WNV 
risultano essere argomenti di grande interesse. 

Le zanzare infette costituiscono il principale vettore di trasmissione del 
virus agli esseri umani. Quando nel 2002 furono riportati i primi casi umani a 
Chicago, il Dipartimento di Sanità Pubblica della città ha avviato un esteso 
programma di sorveglianza e controllo che resta tutt'oggi in vigore.

Dalla fine della primavera all'inizio dell'autunno, numerose trappole per 
zanzare vengono distribuite sul territorio, andando ogni settimana a testare la 
presenza del virus negli esemplari catturati. Sulla base di queste 
informazioni, la città programma l'utilizzo di insetticidi per controllare la 
popolazione di zanzare adulte.

\section{Obiettivi}

Con una media di 91 trappole attive per 14 settimane ogni anno, questo 
programma presenta significativi costi di raccolta dei campioni ed esecuzione 
dei test clinici per determinare la presenza del virus.

% select avg(traps.count)
% from (
%	select season_year, count(distinct trap) FROM "WNV" GROUP BY season_year
% ) as traps

% select avg(min), avg(max), avg(count)
% from (
%	select season_year, trap, min(week), max(week), max(week) - min(week) as 
%	count FROM "WNV" GROUP BY season_year, trap
% ) as weeks

L'obiettivo di questo elaborato è progettare un modello di classificazione 
supervisionata capace di prevedere se ogni settimana verrà rilevato o meno il 
virus in una certa trappola. Le predizioni di questo modello permetteranno di 
indirizzare gli sforzi di raccolta verso quelle trappole in cui ci si aspetta 
di trovare esemplari infetti con probabilità maggiore.

Dato che i test effettuati negli anni sono risultati positivi al virus solo 
nell'8,6\% dei casi, un modello sufficientemente accurato ha le potenzialità di 
ridurre fortemente il numero di trappole da analizzare ogni settimana e di 
conseguenza i costi del programma. 

% 2337 / (2337 + 24860) = 0,08592859507

Si ritene che un clima caldo e secco sia favorevole alla diffusione del WNV. 
Per questo motivo le predizioni si basano principalmente su dati relativi alle 
condizioni meteorologiche e all'ubicazione delle trappole.

\section{Design del dataset}

Il problema descritto è stato proposto dalla città di Chicago sulla piattaforma 
\href{https://www.kaggle.com/c/predict-west-nile-virus}{Kaggle} nel 2015. 
Rispetto a quanto richiesto in quella competizione, per questo progetto abbiamo 
però adottato una strategia leggermente diversa.

Abbiamo potuto constatare che i dati forniti su Kaggle erano già stati oggetto 
di un'elaborazione preliminare e le attività svolte si sovrapponevano 
parzialmente a quanto richiesto per il modulo di Data Technology. Abbiamo 
quindi deciso di non utilizzare questi dataset, ma piuttosto ricostruirli a 
partire dalle loro fonti originali.

Siamo riusciti ad individuare i dati necessari sui portali Open Data del 
Dipartimento di Sanità Pubblica di Chicago (CDPH) e della National Oceanic and 
Atmospheric Administration (NOAA). Come descritto nei 
Capitoli~\ref{chap:data-quality} e \ref{chap:data-integration}, abbiamo poi 
effettuato un'analisi di qualità dei dati grezzi, la correzione delle anomalie 
rilevate e l'integrazione delle due fonti.

Ricostruendo i dataset, abbiamo inoltre potuto ottenere dati più recenti di 
quelli pubblicati su Kaggle, permettendoci di analizzare il periodo 2007--2017 
invece che limitarci agli anni 2007--2013.

Uno svantaggio dell'approccio adottato è che, basandosi su dati differenti, i 
nostri risultati non potranno essere confrontati direttamente con lo stato 
dell'arte visibile sulla classifica della competizione. \textbf{FIXME 
completare questa parte quando abbiamo effettivamente dei risultati}

\section{Eventuali ipotesi o assunzioni}

